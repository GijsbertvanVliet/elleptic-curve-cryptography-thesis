\documentclass[10pt]{article}

\begin{document}
Omdat de berekeningen tot aan (16) (die de gevalsonderscheiding 1 of 2 bepaalt) goed lijken te zijn zal ik hier de berekeningen van case 1 en case 2 apart behandelen:\\
We zullen in de rest van deze tekst de volgende conventie aanhouden:
$$S:=x^3+Ax+B$$

\section{Case 1}
$l$ is een vast gekozen priemgetal. Ook ga ik ervan uit dat $(\frac{q}{l})=1$ en dat $0<w<l$ gekozen is zdd $w^2 \equiv q$ mod $l$.\\
Allereerst moet er gecheckt worden of er een niet nul punt $P \in E[l]$ is zdd $\phi_l(P)=\pm wP$.\\
Dit komt dus neer op het checken dat de x-coordinaten gelijk zijn. Dit ga ik nu doen voor het geval dat $w$ even en oneven is.

\begin{itemize}
\item ($w$ is even.) Als $P=(x,y) \in E[l]$, dan is de x-coordinaat van $\phi_l(P)=x^q$ en de x-coordinaat van $wP=x-\frac{\psi_{w-1}\psi_{w+1}}{\psi _w^2}$. Dus (na reductie van de formules $\psi$:) moeten we volgende vergelijking uitwerken:
$$x^q=x-\frac{\psi'_{w-1}\psi'_{w+1}}{\psi _w^{'2}}$$
Oftewel, met behulp van (6) uit het algorithme:
$$x^q=x-\frac{f_{w-1}f_{w+1}}{y^2f_w^{2}}$$
En dit geeft de formule van (17), waarvan we de de ggd met $f_l$ moeten berekenen.
\item ($w$ is oneven.) Dit komt neer op De volgende vergelijking uitwerken:
$$  x^p=x-\frac{\psi_{w-1}\psi_{w+1}}{\psi _w^2}=x-\frac{\psi'_{w-1}\psi'_{w+1}}{\psi _w^{'2}}=x-\frac{y^2f_{w-1}f_{w+1}}{f_w^{2}} $$
Dit geeft dus precies de tweede vergelijking uit (17).
\end{itemize}

Als deze ggd berekend is en niet gelijk aan 1 is, moet worden gecheckt of $\phi _lP=wP$ voor een $P \in E[l]$. Als dit het geval is, dan geldt dat $t \equiv 2w$ mod $l$ en anders $t \equiv -2w$ mod $l$.\\
Checken of $\phi _lP = wP$ komt neer op checken of de y-coordinaten overeen komen.

\begin{itemize}
\item ($w$ is even.) We moeten de volgende vergelijking uitwerken en de ggd met $f_l$ nemen (wederom zijn de formules $\psi$ al gereduceerd:
$$  y^q=\frac{\psi'_{w+2}\psi_{w-1}^{'2}-\psi'_{w-2}\psi_{w+1}^{'2}}{4y\psi_w^{'3}}=\frac{y(f_{w+2}f_{w-1}^2-f_{w-2}f_{w+1}^2)}{4y^4f_w^3} $$
Dit uitwerken en reduceren geeft dus volgende formule:
$$4S^{\frac{q+3}{2}}f_w^3-f_{w+2}f_{w-1}^2+f_{w-2}f_{w+1}^2$$
waarvan de ggd met $f_l$ genomen moet worden.
\item ($w$ is oneven.) Dit komt neer op het uitwerken van de volgende gelijkheid:
$$  y^q=\frac{\psi'_{w+2}\psi_{w-1}^{'2}-\psi'_{w-2}\psi_{w+1}^{'2}}{4y\psi_w^{'3}}=\frac{y^2(f_{w+2}f_{w-1}^2-f_{w-2}f_{w+1}^2)}{4yf_w^3}$$
Dit uitwerken en reduceren levert de volgende formule:
$$4S^{\frac{q-1}{2}}f_w^3-f_{w+2}f_{w-1}^2+f_{w-2}f_{w+1}^2$$
\end{itemize}

\section{Case 2}

Hier moet gecheckt worden voor welke $0<\tau<l$ geldt dat $\phi_l^2(P)+kP=\tau \phi_l(P)$ voor alle $P \in E[l]$ ($k \equiv q$ mod $l$)\\
Nu geldt voor $P=(x,y)$ dat $$\phi^2_l(P)+kP = (-x^{q^2}-x+\frac{\psi_{k-1}\psi_{k+1}}{\psi_k^2}+\lambda^2,-y^{q^2}-\lambda(-2x^{q^2}-x+\frac{\psi_{k-1}\psi_{k+1}}{\psi_k^2}+\lambda^2))$$
waar
$$ \lambda=\frac{\psi_{k+2}\psi_{k-1}^2-\psi_{k-2}\psi_{k+1}^2-4y^{q^2+1}\psi_k^3}{4\psi_ky((x-x^{q^2})\psi_k^2-\psi_{k-1}\psi_{k+1})}=\frac{\alpha}{\beta} $$
En $\tau \phi_l(P) = (x^q-(\frac{\psi_{\tau+1}\phi_{\tau-1}}{\psi_\tau^2})^q,(\frac{\psi_{\tau+2}\psi_{\tau-1}^2-\psi_{\tau-2}\psi_{\tau+1}^2}{4y\psi_\tau^3})^q)$\\
Vanwege efficientie van het algoritme is het wederom handig om de formules $f$ te gebruiken in plaats van $\psi$.\\
Hieronder volgen de verkregen formules voor $\alpha$ en $\beta$:

\begin{itemize}
\item ($k$ is even) $$\alpha=f_{k+2}f^2_{k-1}-f_{k-2}f^2_{k+1}-4S^{\frac{q^2+3}{2}}f_k^3$$ \\
(Merk op dat hier $\alpha$ door $y$ gedeeld is om op het juiste andwoord te komen.)\\
$$\beta=4Sf_k((x-x^{q^2})Sf_k^2-f_{k-1}f_{k+1})$$
\item ($k$ is oneven) $$\alpha=S(f_{k+2}f^2_{k-1}-f_{k-2}f^2_{k+1})-4S^\frac{q^2+1}{2}f^3_k$$
$$\beta=4f_k((x-x^{q^2})f_k^2-Sf_{k-1}f_{k+1})$$
(Merk op dat hier $\beta$ door $y$ gedeeld is om op het juiste andwoord te komen.)
\end{itemize} 
Er moeten nu dus twee gelijkheden uitgeschreven worden:

\begin{enumerate}
\item $$-x^{q^2}-x+\frac{\psi_{k-1}\psi_{k+1}}{\psi_k^2}+\frac{\alpha^2}{\beta^2}=x^q-(\frac{\psi_{\tau+1}\phi_{\tau-1}}{\psi_\tau^2})^q$$
$$ \Rightarrow -(x^{q^2}+x^q+x)\psi_k^2+\psi_{k-1}\psi_{k+1}+\psi_k^2\frac{\alpha^2}{\beta^2}=-\psi_k^2(\frac{\psi_{\tau+1}\phi_{\tau-1}}{\psi_\tau^2})^q$$
$$ \Rightarrow \psi_\tau^{2q}(\beta^2(-(x^{q^2}+x^q+x)\psi_k^2+\psi_{k-1}\psi_{k+1})+\psi_k^2\alpha^2)+\beta^2\psi_k^2\psi_{\tau+1}^q\psi_{\tau-1}^q=0  $$
Dit polynoom (vanaf nu genaamd $Pol_1$.) wordt genoteerd als $Pol_1=\psi_\tau^{2q}\delta_1+(\psi_{\tau-1}\psi_{\tau+1})^q \delta_2$.\\

Eerst worden $\delta_1$ en $\delta_2$ berekend:\\
(Merk op dat hier voor $k$ is even $y\alpha$ gescheven is in plaats voor $\alpha$ omdat we in de voorgaande berekening door $y$ gedeeld hebben. Hetzelfde geld voor $\beta$ voor het geval dat $k$ oneven is.)
\begin{itemize}
\item ($k$ is even) $$\delta_1=\beta^2(f_{k-1}f_{k+1}-(x^{q^2}+x^q+x)Sf_k^2)+\alpha^2S^2f_k^2$$
\item ($k$ is oneven) $$\delta_1=\beta^2S(Sf_{k-1}f_{k+1}-(x^{q^2}+x^q+x)f_k^2)+\alpha^2f_k^2$$
\end{itemize}
En voor zowel $k$ is even als oneven:$$\delta_2=\beta^2Sf_k^2$$

Vervolgens wordt $Pol_1$ uitgeschreven:
\begin{itemize}
\item ($\tau$ is even) $$Pol_1=f_\tau^{2q}S^q\delta_1+(f_{\tau-1}f_{\tau+1})^q\delta_2$$
\item ($\tau$ is oneven) $$Pol_1=f_\tau^{2q}\delta_1+(f_{\tau-1}f_{\tau+1})^qS^q\delta_2$$
\end{itemize}

En van deze formule moet dus gecheckt worden dat het nul modulo $f_l$ is.
\item $$-y^{q^2}-\frac{\alpha}{\beta}(-2x^{q^2}-x+\frac{\psi_{k-1}\psi_{k+1}}{\psi_k^2}+\frac{\alpha^2}{\beta^2})=(\frac{\psi_{\tau+2}\psi_{\tau-1}^2-\psi_{\tau-2}\psi_{\tau+1}^2}{4y\psi_\tau^3})^q$$
$$\Rightarrow 4y^q\psi_\tau^{3q}(-\beta^3y^{q^2}-\alpha(\beta^2(-2x^{q^2}-x+\frac{\psi_{k-1}+\psi_{k+1}}{\psi_k^2})+\alpha^2))=\beta^3(\psi_{\tau+2}\psi_{\tau-1}^2-\psi_{\tau-2}\psi_{\tau+1}^2)^q$$
$$\Rightarrow 4y^q\psi_\tau^{3q}(-\beta^3\psi_k^2y^{q^2}-\alpha\beta^2(\psi_k^2(-2x^{q^2}-x)+\psi_{k-1}\psi_{k+1})+\alpha^3\psi_k^2)-\beta^3\psi_k^2(\psi_{\tau+2}\psi_{\tau-1}^2-\psi_{\tau-2}\psi_{\tau+1}^2)^q=0$$
En deze formule kan iets mooier geschreven worden als
$$ 4y^q\psi_\tau^{3q}(\alpha\beta^2(\psi_k^2(2x^{q^2}+x)-\psi_{k-1}\psi_{k+1})-\psi_k^2(\alpha^3+\beta^3y^{q^2}))-\beta^3\psi_k^2(\psi_{\tau+2}\psi_{\tau-1}^2-\psi_{\tau-2}\psi_{\tau+1}^2)^q $$
Dit polynoom wordt vanaf nu geschreven als $$Pol_2=\psi_\tau^{3q}\delta_3-(\psi_{\tau+2}\psi_{\tau-1}^2-\psi_{\tau-2}\psi_{\tau+1}^2)^q\delta_4$$
Eerst worden $\delta_3$ en $\delta_4$ berekend:

\begin{itemize}
\item ($k$ is even) $$\delta_3=4y^q(y\alpha\beta^2(y^2f_k^2(2x^{q^2}+x)-f_{k-1}f_{k+1})-y^5\alpha^3f_k^2-y^{q^2+2}\beta^3f_k^2)$$
$$=4S^\frac{q+1}{2}(\alpha\beta^2(Sf_k^2(2x^{q^2}+x)-f_{k-1}f_{k+1})-S^2f_k^2(\alpha^3+S^\frac{q^2-3}{2}\beta^3))$$
\item ($k$ is oneven) $$\delta_3=4y^q(y^2\alpha\beta^2(f_k^2(2x^{q^2}+x)-y^2f_{k-1}f_{k+1})-f_k^2(\alpha^3+y^{q^2+3}\beta^3))$$
$$=4S^\frac{q-1}{2}(S\alpha\beta^2(f_k^2(2x^{q^2}+x)-Sf_{k-1}f_{k+1})-f_k^2(\alpha^3+S^\frac{q^2+3}{2}\beta^3))$$
\end{itemize}
En in beide gevallen geldt dat $$\delta_4=\beta^3Sf_k^2$$

Nu hoeft dus alleen nog maar $Pol_2$ uitgewerkt te worden:
\begin{itemize}
\item ($\tau$ is even) $$Pol_2=f_\tau^{3q}S^\frac{3q+1}{2}\delta_3-(f_{\tau+2}f_{\tau-1}^2-f_{\tau-2}f_{\tau+1}^2)^qS^\frac{q+1}{2}\delta_4$$
(Dit polynoom is verkregen door een directe berekening en het geheel vervolgens te delen door $y$.)
\item ($\tau$ is oneven) $$Pol_2=f_\tau^{3q}\delta_3-(f_{\tau+2}f_{\tau-1}^2-f_{\tau-2}f_{\tau+1}^2)^qS^q\delta_4$$
\end{itemize}
En dit is dus het polynoom waarvan gecheckt moet worden dat het nul modulo $f_l$ is.
\end{enumerate}

\end{document}
~                             
