\documentclass{article}
\usepackage{fullpage}
\usepackage{amsmath,amssymb}


\begin{document}
\title{Outline scription}
\author{Gijsbert van Vliet}
\date{\today}
\maketitle

\section{Introduction}
Here I will explain the necessary mathematical background which will be:
\begin{itemize} 
\item Some basic knowledge of algebra. (the reader should be familiar with groups/ rings/ finite fields ect.) 
\item Some knowledge of algebraic geometry. For example, reading and understanding the first two chapters of Silverman's "The arithmetic of Elliptic Curves" will suffice.
\item Just general mathematical maturity. For example I will use the chinese remainder theorem without stating it.
\end{itemize}

\section{Introduction to cryptography}
In this chapter I will give the general outlay of Public-key cryptography and the DLP.
\subsection{Public-key cryptography}
\subsection{complexity}
\subsection{the discrete logarithm problem}
\subsection{General attacks on the DLP}
\subsubsection{The Pohlig-Hellman Method}
\subsubsection{The baby step-giant step algorithm}
\subsubsection{the Pollard $\rho$ and $\lambda$ method}

\section{Elliptic Curves}
In this section I will give the basic theory of elliptic curves. Most proofs will be omitted. For those I will refer to the book of Silverman.\\
Standard notation:\\
-$E/K$: E is defined over K.\\
-$\bar{K}(C)$: the function field of E over $\bar{K}$.\\
-$K(C)$: the function field of E over $K$.
\subsection{Weierstrass Equations}
-Define general/simplified Weierstrass equations.\\
-Define discriminant $\Delta$,j-invariant $j$ and the invariant differential $\omega$.\\
-Proposition 1.4.\\
-Remark: Curves given by a Weierstrass equation are curves of genus 1.
\subsection{Group law}
-Definition of the group law.\\
-Proposition: The group law makes $E$ into an abelian group.\\
-Proposition: The $K$ rational points form a subgroup of E.\\
-Remark: $P=(x,y) \in E \Rightarrow -P=(x,-y)$.\\
-Notation $[m]P$.
\subsection{Elliptic Curves}
-Define Elliptic Curve $E$.\\
-Proposition 3.1. (Possibly with outlay of the proof.)\\
-Example.\\
-Remark: Using RR to give group law on any elliptic curve. Proposition 3.4/3.6.\\
-Corollary 3.5.
\subsection{Isogenies}
-Define Isogeny $\phi$ for elliptic curves.\\
-Remark: Either $\phi(E_1) = \{O\}$ or $\phi(E_1)=E2$.\\
-Define (in)separable degree of $\phi$.\\
-Define Endomorphism and Automorphism ring.\\
-Example 4.1.\\
-Proposition 4.2.(a)\\
-Define $E[m]$.\\
-Explane complex multiplication.\\
-Example(s).\\
-Theorem 4.8.\\
-Remark: Corollary 4.9.\\
-Theorem 4.10.
\subsection{The Frobenius morphism}
-Define the q-th Frobenius morphism for Curve of Characteristic p.\\
-Example.\\
-Proposition II.2.11.\\
-Corollary II.2.12.\\
-Proposition 1.5.\\
-Proposition 5.1.\\
-Theorem 5.2.\\
-Corollary 5.3.\\
-Corollary 5.5. (with prooof)
\subsection{The dual isogeny}
-Theorem 6.1. (a)\\
-Define the dual isogeny.\\
-Theorem 6.2.\\
-Corollary 6.4.
\subsection{The Tate module}
-introductory remarks.\\
-Define Tate module.\\
-Proposition 7.1.\\
-Define the $l$-adic representation of the Galois group.\\
-Theorem 7.4.
\subsection{The Weil Pairing}
-Justification construction.\\
-Construction.\\
-Proposition 8.1.\\
-Corollary 8.1.1.\\
-Proposition 8.2.\\
-Proposition 8.3.\\
-Proposition 8.6.
\subsection{the endomorphism ring}
-Define complex quadratic order.\\
-Define quaternion algebra.\\
-Corollary 9.2.\\
-Remark: endomorphism ring of elliptic curve over finite field is never $\mathbb{Z}$.

\section{Elliptic Curves over Finite Fields}
Standard notation:\\
-q is a power of a prime p.\\
-$\mathbb{F}_q$ is a finite field with q elements.\\
-$\bar{\mathbb{F}}_q$ is an algebraic Closure of $\mathbb{F}_q$.\\
-Remark ECDLP.\\
-Give small improvement for baby step- giant step algorithm.
\subsection{The number of rational points}
-Rough estimate $\leq 2q+1$.\\
-Hasse's theorem.\\
-Use of legendre symbol to calculate number of rational points.\\
-Define Trace of frobenius morphism.\\
-Theorem 2.3.1.\\
-Linear recurrence for trace of $q^n$-frobenius morphism.
\subsection{The endomorphism ring}
-Theorem 3.1.(With proof)\\
-definition supersingular, ordinary and Hasse invariant.\\
-Theorem: E is supersingular iff p divides the trace of the frobenius morphism.
\subsection{The group structure of elliptic Curves}
-Proposition: type of elliptic curve over finite field.\\
-Lemma1 from article MOV-attack.\\
-Lemma2 from article MOV-attack.\\
-Lemma3 from article MOV-attack.
\subsection{Determining the Hasse Invariant}
-Theorem 4.1.\\
-examples.

\section{Schoof's algorithm}
\subsection{The division polynomials}
-Define division polynomials $\psi_n$, the polynomials $f_n$ and give recurrence relations.\\
-Proposition 2.1 from Schoof's Algorithm.\\
-Proposition 2.2 from Schoof's Algorithm.
\subsection{General outlay of the algorithm}
-give relation determining the trace of frobenius mod $l$.\\
-Explain Estimation for number of primes.\\
-Give the relations to be tested.
\subsection{Detailed description of the algorithm}
-Give definition of gcd determining case distinction.\\
-Give detailed description of case 1.\\
-Give detailed description of case 2.
\subsection{Efficiency of Schoof's algorithm}
-Give theoretical complexity of algorithm.\\
-Give examined running times for several different primes.\\
-Draw conclusion.\\
-Remark on existing improvements of algorithm.

\section{The MOV Attack}}
-Introductory remarks.
\subsection{Index calculus}
-General outlay.\\
-Worked out example.
\subsection{Calculating the Weil Pairing}
-description of algorithm.\\
-remark on complexity of the algorithm.
\subsection{The Reduction}
(The referrals are here to the article about the MOV attack.)\\
-Lemma 4.\\
-Lemma 5.\\
-Example.\\
-Lemma 6.\\
-Lemma 7.\\
-Theorem 10.\\
-Give algorithm.
\subsection{Supersingular Curves}
-Give Table 1 about supersingular curves.\\
-Give example(s) about how information of Table 1 was obtained.\\
-Give algorithm.\\
-Remarks on why the algorithm works.
\subsection{Complexity}
-Give theoretical estimation of the complexity of the algorithm.\\
-(If this algorithm is actually implemented) give examined running times and draw appropriate conclusions. 

\section{Anomalous curves}
-Define Anomalous curves.\\
-Example.
\subsection{The case $p=q$.}
\subsection{The general case.}

\section{Elliptic Curve cryptography}
\subsection{Diffie-Hellman Key Exchange}
\subsection{Massey-Omura Encryption}
\subsection{ElGamal Public Key Encryption}
\subsection{A cryptosystem based on the Weil Pairing}



\end{document}
