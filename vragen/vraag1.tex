\documentclass{article}
\usepackage{amsmath,amssymb}
\usepackage{tikz-cd} 


\begin{document}
\title{Vragen algorithme Ren\'e Schoof}
\author{Gijsbert van Vliet}
\date{\today}
\maketitle

\begin{itemize}
\item In het algorithme van Ren\'e Schoof wordt een afbeelding $$(1) End_{\mathbb{F}_q}E \rightarrow End_{Gal(\overline{\mathbb{F}}_q/\mathbb{F}_q)} E[l]$$ gedefinieerd.\\
Ik weet dat $End_{\mathbb{F}_q}E$ precies de endomorphismen\\ $\phi$ van $E$ zijn zodanig dat voor iedere $\sigma \in Gal(\overline{\mathbb{F}}_q/\mathbb{F}_q)$ geldt dat $$\phi(P^\sigma) = \phi(P)^\sigma$$
Is het nu ook zo dat $End_{Gal(\overline{\mathbb{F}}_q/\mathbb{F}_q)} E[l]$ precies de endomorphismen $\phi: E[l] \rightarrow E[l]$ zijn die commuteren met ieder element $\sigma \in Gal(\overline{\mathbb{F}}_q/\mathbb{F}_q)$ en dat afbeelding (1) simpelweg gegeven is door restrictie van endomorphismen van $E$ tot endomorphismen van $E[l]$?
\item Verderop in het algorithme wordt een onderscheid gemaakt tussen twee gevallen. Het eerste geval is het geval dat er een $P \in E[l]$ bestaat met $\phi_l^2P = \pm qP$.\\
In het speciale geval dat $\phi_l^2P = qP$ wordt in \'e\'en zin de conclussie getrokken dat als $\left(\frac{q}{l}\right)=-1$, dat dan $t \equiv 0$ (mod $l$). Ik zie niet in waarom deze conclussie zo snel getrokken kan worden.\\
Ook wordt in dit gedeelte van het algorithme gesproken over de eigenwaarde van $\phi_l$. Is dit gewoon de eigenwaarde van $\phi_l$ waarbij deze ge\"intepreteerd wordt als element van $Gl_2(\mathbb{Z}_l)$?
\end{itemize}

\end{document}